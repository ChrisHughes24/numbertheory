\def\paperversiondraft{draft}
\def\paperversionblind{blind}

\ifx\paperversion\paperversionblind
\else
  \def\paperversion{blind}
\fi

% !TEX TS-program = pdflatex
% !TEX encoding = UTF-8 Unicode

% This is a simple template for a LaTeX document using the "article" class.
% See "book", "report", "letter" for other types of document.

\documentclass[11pt]{article} % use larger type; default would be 10pt

\usepackage{xargs}
\usepackage{todonotes}
\usepackage{xparse}
\usepackage{xifthen, xstring}
\usepackage{ulem}
\usepackage{xspace}
\makeatletter
\font\uwavefont=lasyb10 scaled 652

\newcommand\colorwave[1][blue]{\bgroup\markoverwith{\lower3\p@\hbox{\uwavefont\textcolor{#1}{\char58}}}\ULon}
% \makeatother

% \ifx\paperversion\paperversiondraft
% \newcommand\createtodoauthor[2]{%
% \def\tmpdefault{emptystring}
% \expandafter\newcommand\csname #1\endcsname[2][\tmpdefault]{\def\tmp{##1}\ifthenelse{\equal{\tmp}{\tmpdefault}}
%    {\todo[linecolor=#2!20,backgroundcolor=#2!25,bordercolor=#2,size=\tiny]{\textbf{#1:} ##2}}
%    {\ifthenelse{\equal{##2}{}}{\colorwave[#2]{##1}\xspace}{\todo[linecolor=#2!10,backgroundcolor=#2!25,bordercolor=#2]{\tiny \textbf{#1:} ##2}\colorwave[#2]{##1}}}}}
% \else
% \newcommand\createtodoauthor[2]{%
% \expandafter\newcommand\csname #1\endcsname[2][\@nil]{}}
% \fi


% \createtodoauthor{chris}{red}
% \createtodoauthor{tobias}{blue}
% \createtodoauthor{authorThree}{green}
% \createtodoauthor{authorFour}{orange}
% \createtodoauthor{authorFive}{purple}


%%% Examples of Article customizations
% These packages are optional, depending whether you want the features they provide.
% See the LaTeX Companion or other references for full information.

%%% PAGE DIMENSIONS
\usepackage{geometry} % to change the page dimensions
\geometry{a4paper} % or letterpaper (US) or a5paper or....
\geometry{margin=1in} % for example, change the margins to 2 inches all round
% \geometry{landscape} % set up the page for landscape
%   read geometry.pdf for detailed page layout information

\usepackage{graphicx} % support the \includegraphics command and options

% \usepackage[parfill]{parskip} % Activate to begin paragraphs with an empty line rather than an indent

% Broaden margins to make room for todo notes
\ifx\paperversion\paperversiondraft
  \paperwidth=\dimexpr \paperwidth + 6cm\relax
  \oddsidemargin=\dimexpr\oddsidemargin + 3cm\relax
  \evensidemargin=\dimexpr\evensidemargin + 3cm\relax
  \marginparwidth=\dimexpr \marginparwidth + 3cm\relax
  \setlength{\marginparwidth}{4cm}
\fi

\usepackage[utf8x]{inputenc}
\usepackage{amssymb}

\usepackage{color}
\definecolor{keywordcolor}{rgb}{0.7, 0.1, 0.1}   % red
\definecolor{commentcolor}{rgb}{0.4, 0.4, 0.4}   % grey
\definecolor{symbolcolor}{rgb}{0.0, 0.1, 0.6}    % blue
\definecolor{sortcolor}{rgb}{0.1, 0.5, 0.1}      % green

\usepackage{listings}


%%% PACKAGES
\usepackage{inputenc}
\usepackage{booktabs} % for much better looking tables
\usepackage{array} % for better arrays (eg matrices) in maths
\usepackage{paralist} % very flexible & customisable lists (eg. enumerate/itemize, etc.)
\usepackage{verbatim} % adds environment for commenting out blocks of text & for better verbatim
\usepackage{subfig} % make it possible to include more than one captioned figure/table in a single float

\usepackage{textcomp}


% These packages are all incorporated in the memoir class to one degree or another...

%%% HEADERS & FOOTERS
\usepackage{fancyhdr} % This should be set AFTER setting up the page geometry
\pagestyle{fancy} % options: empty , plain , fancy
\renewcommand{\headrulewidth}{0pt} % customise the layout...
\lhead{}\chead{}\rhead{}
\lfoot{}\cfoot{\thepage}\rfoot{}

%%% SECTION TITLE APPEARANCE
\usepackage{sectsty}
\allsectionsfont{\sffamily\mdseries\upshape} % (See the fntguide.pdf for font help)
% (This matches ConTeXt defaults)

%%% ToC (table of contents) APPEARANCE
\usepackage[nottoc,notlof,notlot]{tocbibind} % Put the bibliography in the ToC
\usepackage[titles,subfigure]{tocloft} % Alter the style of the Table of Contents
\renewcommand{\cftsecfont}{\rmfamily\mdseries\upshape}
\renewcommand{\cftsecpagefont}{\rmfamily\mdseries\upshape} % No bold!

\setlength{\parindent}{0em}
\setlength{\parskip}{1em}
\usepackage{amsmath}
\usepackage{upgreek}
\newtheorem{lemma}{Lemma}
\newtheorem{sublemma}{Lemma}[lemma]


%%% END Article customizations

%%% The "real" document content comes below...

\title{Number Theory Coursework}
\date{}


\begin{document}
\maketitle

\section{Preliminaries}

For this proof one related theorem is required. For any $n \in \mathbb{N}_{\ge0}$, let
$v_p(n)$ be the maximum integer $k$ such that $p^k$ divides $n$. Then for any integers $n > 0$, and $k > 0$,
$v_p{{n}\choose{k}} \ge v_p(n) - v_p(k)$.

A theorem by Legendre says that for any $n$, $v_p(n!) = \sum_{i=0}^{\infty} \left\lfloor\frac{n}{p^i}\right\rfloor $.

To compute a bound on $v_p{{n}\choose{k}}$, observe that
\begin{equation}
  \begin{aligned}
    v_p{{n}\choose{k}} = v_p(n!) - v_p(k!) - v_p ((n-k)!) \\
    = \sum_{i=0}^{\infty} \left\lfloor\frac{n}{p^i}\right\rfloor - \left\lfloor\frac{k}{p^i}\right\rfloor
      - \left\lfloor\frac{n-k}{p^i}\right\rfloor
  \end{aligned}
\end{equation}

Observe that if $p^i$ divides $n$ but not $k$ then the corresponding term in the sum is $1$,
so the sum is at least as big as the number if $i$ that divide $n$ but not $k$, which is
$v_p(n) - v_p(k)$

\section{Main proof}
\begin{lemma}\label{lem:even2}
  If $m$ is even then $x+1$ is a power of $2$
\end{lemma}
Consider congruences mod $p$ when $m$ is even, and $p$ divides $x+1$
\begin{equation}
  x^m + 1 \equiv (-1)^m + 1 \equiv 2 \ \text{mod} \ p
\end{equation}
So $2$ divides $(x+1)^n = 0$ mod $p$ and therefore $p = 2$ and $x + 1$ is a power of $2$.
\begin{lemma}\label{lem:2mod4}
  If $m$ is even then $x^m + 1 \equiv 2 \ mod \ 4$
\end{lemma}
  This is because $x$ must be odd, so $x^m$ is an odd square, so $x^m$ is $1 \ \text{mod} \ 4$.
\begin{lemma}
  $m$ is odd
\end{lemma}
Assume $m$ is even.

By Lemma \ref{lem:even2}, $x^m + 1$ is a power of $2$. But by
Lemma \ref{lem:2mod4} $x^m + 1$ is $2$ mod $4$. $2$ is the only power of $2$ congruent to
$2$ mod $4$. So $x^m + 1 = 2$, but this contradicts $x > 1$ and $m > 1$. So $m$ is odd.

\begin{lemma}\label{lem:poly}
$x + 1$ divides $x^m + 1$
\end{lemma}
$-1$ is a root of the polynomial $X^m + 1$ since $m$ is odd. So by the factor theorem $x+1$ divides
$x^m + 1$.

The idea of the rest of the proof is to prove that $\frac{x^m+1}{x+1}$ divides $m$, and is therefore less
than or equal to $m$, which puts bounds on $x$ and $m$. The divisibility is proven by proving every prime power that
dividies $\frac{x^m+1}{x+1}$ also divides $m$.

\begin{lemma}\label{lem:dvdchoose}
  Let $p$ be a prime and suppose $p^r$ divides $m$. Let $t$ be a positive integer.
  Let $i$ be an integer at least $2$. The $p^{r + t + 1}$ divides ${{m}\choose{i}} p^{ti}$
\end{lemma}

If $p=2$ then since $m$ is odd, $r = 0$. So $p^{r + t +1} = p^{t+1}$ which divides $p^{ti}$.

Otherwise, we use the fact that $v_p{{m}\choose {i}} \ge v_p(m) - v_p(i)$.

First prove that $p^{ti - t} > i$.
$p^{ti - t} = p ^{t} p^{t(i-2)} \ge 3p^{t(i-2)} \ge p^{t(i-2)} + 2 > t(i-2) + 2 \ge i$,
So $p^{ti - t}$ does not divide $i$.

Therefore $v_p{{m}\choose {i}} > v_p(m) - (ti - t) $.
So $v_p\left({{m}\choose {i}}p^{ti}\right) \ge v_p(m) + t$.

\begin{lemma}\label{lem:primepowdvd}
  If $p$ is prime and $p^t$ divides $x+1$, but $p^{t+1}$ does not divide $x+1$ and
  $p^{s+t}$ divides $x^m + 1$, then $p^{s}$ divides $m$.
\end{lemma}

If $t=0$ then $s$ must also be equal to zero, since any prime dividing $x^m+1$ also divides $x+1$.

First prove that $p^r$ divides $m$, when $r \le s$ by induction on $r$.
We can assume $t$ is positive.

The case $r = 0$ is trivial.

Now assume $p^r$ divides $m$ and deduce $p^{r+1}$ divides $m$, provided $r < s$.


Write $x = kp^t - 1$ where $p$ does not divide $k$.

Then $x^m+1 = (kp^t-1)^m + 1 = mkp^t- \sum_{i=2}^{m} {{m}\choose{i}} (-k)^i p^{it}$.

Using Lemma \ref{lem:dvdchoose}, $p^{r + t + 1}$ divides the sum. It also divides $x^m+1$, so
$p^{r+ t + 1}$ must divide $mkp^t$, so $p^{r+1}$ divides m.

\begin{lemma}\label{lem:ltm}
  $\frac{x^m+1}{x+1} \le m$
\end{lemma}

By Lemma \ref{lem:primepowdvd}, every prime power that divides $\frac{x^m+1}{x+1}$ also divides $m$,
So $\frac{x^m+1}{x+1}$ divides $m$ and is therefore less than or equal to $m$.

\begin{lemma}
  m=3
\end{lemma}
Suppose for a contradiction that $m \ge 4$. Write $m = m' + 4$, where $m' \in \mathbb{N}_0$.
Similarly write $x = x' + 2$.

Now show $(m' + 4) (x' + 2) + m' + 4 < (x' + 2)^{m' + 4} + 1$.
\begin{equation}
  \begin{aligned}
    (m' + 4) (x' + 2) + x'(m' + 4) \\
    = (3m'+12) + x'(m'+4) \\
    < (3(x' + 2)^{m'} + 12(x'+2)^{m'}) + x'((x' + 2)^{m'} + 4(x'+2)^{m'}) \\
    \le (x' + 2)^{m' + 4} + 1
  \end{aligned}
\end{equation}
But by Lemma \ref{lem:ltm}, $(m' + 4) (x' + 2) + m' + 4 \ge (x' + 2)^{m' + 4} + 1$,
so $m$ cannot be greater than $3$. $m$ is odd and $m > 1$ so $m = 3$

\begin{lemma}
  x=2
\end{lemma}
Suppose for a contradiction that $x \ge 3$. Write $x = x' = 3$
Then
\begin{equation}
  \begin{aligned}
    mx + m  \\
    = 3x' + 12  \\
    < {x'}^3+9{x'}^2 + 27{x'} + 27 \\
    = x^m+1
  \end{aligned}
\end{equation}
So $x < 3$, but $x > 1$, so $x = 2$.

\begin{lemma}
  The solutions are $x=2$, $m=3$ and $n \ge 2$
\end{lemma}
Given $x=2$ and $m=3$ simple calculations verify that this is a solution if and only if $n\ge2$.

\end{document}
